%%%%%%% INIZIO PREAMBOLO %%%%%%%

\documentclass[12pt, letterpaper]{article}
\usepackage[utf8]{inputenc}
\usepackage[T1]{fontenc}
\usepackage{graphicx}
\usepackage{titling}
\usepackage{geometry}
\usepackage{imakeidx}
\usepackage[italian]{babel}
\makeindex
\usepackage{listings}
\usepackage{xcolor}
\usepackage{caption}
\usepackage{hyperref}
\usepackage{fancyhdr}
\usepackage{longtable}
\DeclareCaptionFormat{myformat}{\textbf{#1#2}#3} % Formato personalizzato
\captionsetup[lstlisting]{format=myformat, labelsep=space}
\renewcommand{\lstlistingname}{Codice} 

%Intestazione e pie' di pagina
\pagestyle{fancy}
\lhead{}
\chead{Fondamenti d'intelligenza artificiale}
\rhead{13/02/2025}
\lfoot{Brigid}
\cfoot{\thepage}
\rfoot{Esame 2025}

\lstset{
	basicstyle=\ttfamily\normalsize,    % Font monospace con dimensione normale
	keywordstyle=\color{teal}\bfseries, % Parole chiave in teal e grassetto
	commentstyle=\color{gray}\itshape,  % Commenti in grigio e corsivo
	stringstyle=\color{orange},         % Stringhe in arancione
	numbers=left,                       % Numerazione delle righe a sinistra
	numberstyle=\small\color{gray},     % Numeri di riga piccoli e grigi
	stepnumber=1,                       % Numerazione per ogni riga
	breaklines=true,                    % A capo automatico per righe lunghe
	frame=rounded,                      % Cornice arrotondata intorno al codice
	rulecolor=\color{lightgray},        % Colore del bordo
	backgroundcolor=\color{gray!10},    % Sfondo chiaro
	captionpos=b,                       % Posizione della didascalia (sotto il codice)
	showstringspaces=false,             % Non mostra spazi nelle stringhe
	tabsize=4,                          % Dimensione dei tabulatori
	xleftmargin=20pt,                   % Margine interno a sinistra
	xrightmargin=20pt,                  % Margine interno a destra
}

% Imposta i margini della pagina
\geometry{top=2cm, bottom=2cm, left=2.5cm, right=2.5cm}

% Dati per il titolo
\title{\Huge \textbf{Documentazione}\\[0.5cm]
\Large \textbf{Progetto Brigid AI Chatbot}}
\author{\large Catello Martone, Davide Viola, Gabriella Fede, Pasquale Anatriello}

%%%%%%% FINE PREAMBOLO %%%%%%%


\begin{document}
	
% Aggiungi il logo (opzionale)
\begin{figure}[t]
\centering
\includegraphics[width=0.3\textwidth]{brigid.png}
\end{figure}
	
% Genera il titolo
\maketitle
\vfill
% Informazioni aggiuntive
\begin{center}
Prof. Fabio Palomba, Corso di Fondamenti di Intelligenza Artificiale\\
Università degli Studi di Salerno, A.A. 2024/2025
\end{center}
	
\newpage
\tableofcontents
\newpage
	
\section{Introduzione}
	
\subsection{L'obiettivo della chatbot Brigid}
L'obiettivo della chatbot \textbf{Brigid} è quello di fornire un supporto psicologico, incluso all'interno del progetto \textit{Diversify}. La chatbot offre un sostegno concreto e monitora l'andamento delle conversazioni, raccogliendo dati attraverso l'interazione con l'utente.\\ 
Questi dati saranno utilizzati per ottimizzare il funzionamento della chatbot, adattando le conversazioni alle esigenze e alle preferenze specifiche di ciascun utente. In questo modo, l'utente potrà vivere un'esperienza interattiva più personalizzata e coinvolgente, con risposte e contenuti che riflettono in maniera più precisa le sue aspettative e necessità.
	
\section{Descrizione dell'agente}
	
\subsection{Obiettivi}
Un agente chatbot è un sistema basato sull'intelligenza artificiale, progettato per interagire con gli utenti attraverso conversazioni simulate in linguaggio naturale. È in grado di rispondere a domande relative al supporto psicologico. Questo agente è programmato per comprendere input testuali, elaborare le richieste e fornire risposte pertinenti, coerenti e contestualizzate.
La chatbot opera attraverso due modelli distinti: il primo, personalizzato, si focalizza sul fornire supporto emotivo; il secondo, \textit{Gemini}, genera risposte fuori contesto, ampliando così la gamma delle interazioni possibili.

\subsection{Specifica PEAS}
\subsection{Analisi del problema}
La creazione di una chatbot per il supporto psicologico presenta diverse complessità. Comprendere il linguaggio umano, spesso ambiguo o emotivamente carico, richiede modelli avanzati di elaborazione del linguaggio naturale. Inoltre, generare risposte empatiche e contestuali è una sfida, poiché la chatbot deve bilanciare personalizzazione e coerenza, evitando risposte inappropriate o influenzate da bias. Dal punto di vista tecnico, il sistema deve essere scalabile per gestire molteplici conversazioni senza ritardi. \\
Abbiamo affrontato queste sfide adottando modelli NLP avanzati, come GPT, personalizzati per riconoscere emozioni e intenzioni. Per garantire risposte empatiche e sicure, abbiamo combinato risposte generative con risposte predefinite per situazioni delicate. L’architettura serverless assicura scalabilità e rapidità.\\ Queste soluzioni consentono alla chatbot di offrire un supporto psicologico efficace e adattato alle esigenze individuali.

\section{Raccolta, analisi e preprocessing dei dati}

\subsection{Scelta del dataset}
Abbiamo iniziato utilizzando un dataset in formato JSON trovato online sul sito Kaggle, che poi abbiamo ampliato e perfezionato attraverso un lavoro mirato, avvalendoci anche del supporto di ChatGPT per integrare ulteriori dati. Questo processo ci ha permesso di sviluppare un dataset altamente specifico e perfettamente adattato alle esigenze della chatbot. Grazie a questa personalizzazione, siamo riusciti a garantire un’esperienza più efficace e rilevante per gli utenti che interagiscono con il sistema.\\
I dati presenti nel dataset di Kaggle seguono una struttura composta da tag, pattern e response. Questo ci ha permesso di mantenere una coerenza strutturale, applicando la stessa organizzazione anche ai dati aggiunti successivamente durante il nostro lavoro.\\
Il dataset in questione è reperibile a questo \href{https://www.kaggle.com/code/jocelyndumlao/chatbot-for-mental-health-conversations}{\textcolor{blue}{\underline{link}}}.

\subsection{Tecnologie utilizzate}
Abbiamo utilizzato come ambiente di sviluppo PyCharm, che ci ha permesso di utilizzare come linguaggio di programmazione Python. 

\subsubsection{Librerie utilizzate}
Le librerie che abbiamo utilizzato sono:
\begin{itemize}
	\item \textbf{Numpy}: Per manipolazione di array numerici, utilizzati da Bert.
	\begin{lstlisting}[language=Python, caption=Utilizzo di Numpy nel codice]
	# Conversione degli embedding e delle etichette in array NumPy:
	train_x = np.array(train_x)
	train_y = np.array(train_y)
	
	#Calcolo degli argmax per le previsioni e le etichette vere:
	y_pred_classes = np.argmax(y_pred, axis=1)
	y_true_classes = np.argmax(train_y, axis=1)
	
	#Calcolo delle confidence scores:
	confidences = np.max(y_pred, axis=1)
	
	#Conversione degli embedding BERT in array NumPy:
	return outputs.last_hidden_state.mean(dim=1).squeeze().numpy()
\end{lstlisting}
	\item \textbf{JSON}: Per leggere il file intents.json (contiene i pattern e i tag associati).
	\begin{lstlisting}[language=Python, caption=esempio breve di file JSON]
	"tag": "sad",
	"patterns": [
	"I feel sad",
	"I am not feeling good",
	"I feel down today",
	"I am feeling low",
	"I am upset",
	"I feel so lonely"
	],
	"responses": [
	"I am sorry you are feeling this way. It is okay to be sad sometimes.",
	"Take your time to process these feelings. You are not alone.",
	"It is important to be kind to yourself during tough moments.",
	"I am here to listen if you need someone to talk to."
	]
\end{lstlisting}
	
	\item \textbf{Pickle}: Per salvare oggetti Python (come le classi) su disco.
	\begin{lstlisting}[language=Python, caption=Utilizzo di Pickle nel codice]
# Salva le classi e i dati
	pickle.dump(classes, open('../models/classes.pkl', 'wb'))
\end{lstlisting}
	\item \textbf{Transformers} (da Hugging Face): Per caricare BERT e il suo tokenizer.
	\begin{lstlisting}[language=Python, caption=Utilizzo di Transformers nel codice]
#Caricamento del Tokenizer e del Modello BERT
	from transformers import BertTokenizer, BertModel
	
	# Carica il tokenizer e il modello pre-addestrato di BERT
	tokenizer = BertTokenizer.from_pretrained('bert-base-uncased')
	bert_model = BertModel.from_pretrained('bert-base-uncased')
	
#Funzione per ottenere l'Embedding con BERT
	def get_bert_embedding(sentence):
	inputs = tokenizer(sentence, return_tensors="pt", padding=True, truncation=True, max_length=50)
	with torch.no_grad():
	outputs = bert_model(**inputs)
	# Usa la media degli hidden states per rappresentare la frase
	return outputs.last_hidden_state.mean(dim=1).squeeze().numpy()
	
#Creazione dei Dati di Addestramento
	train_x = []
	train_y = []
	
	for pattern, tag in documents:
	# Ottieni l'embedding per la frase
	embedding = get_bert_embedding(pattern)
	train_x.append(embedding)
	
	# One-hot encoding del tag
	output_row = [0] * len(classes)
	output_row[classes.index(tag)] = 1
	train_y.append(output_row)
	
	train_x = np.array(train_x)
	train_y = np.array(train_y)
\end{lstlisting}
	\item \textbf{PyTorch}: Utilizzato per il modello BERT.
	\begin{lstlisting}[language=Python, caption=Utilizzo di PyTorch nel codice]
#Tokenizzazione e Inferenza con BERT:
	inputs = tokenizer(sentence, return_tensors="pt", padding=True, truncation=True, max_length=50)
	with torch.no_grad():
	outputs = bert_model(**inputs)
	
#Estrazione degli embedding di BERT:
	return outputs.last_hidden_state.mean(dim=1).squeeze().numpy()
\end{lstlisting}
	\item \textbf{codice/TensorFlow/Keras}: Per costruire e addestrare il modello di classificazione.
	\begin{lstlisting}[language=Python, caption=Utilizzo di TensorFlow e Keras nel codice]
	#Costruzione del modello di rete neurale
	from tensorflow.keras.models import Sequential
	from tensorflow.keras.layers import Dense, Dropout
	from tensorflow.keras.optimizers import Adam
	from tensorflow.keras.metrics import Precision, Recall
	
	# Costruzione del modello di classificazione
	model = Sequential()
	model.add(Dense(128, input_shape=(train_x.shape[1],), activation='relu'))
	model.add(Dropout(0.5))
	model.add(Dense(64, activation='relu'))
	model.add(Dropout(0.5))
	model.add(Dense(len(train_y[0]), activation='softmax'))
	
	# Compilazione del modello
	model.compile(
	optimizer=Adam(learning_rate=0.001),
	loss='categorical_crossentropy',
	metrics=['accuracy', Precision(), Recall()]
	)
	
	#Addestramento del modello 
	hist = model.fit(train_x, train_y, epochs=100, batch_size=8, verbose=1)
	
	#Salvataggio del modello
	# Salva il modello addestrato
	model.save('../models/chatbot_model.keras')
	
	#Previsioni con il modello addestrato
	y_pred = model.predict(train_x)
	y_pred_classes = np.argmax(y_pred, axis=1)
	y_true_classes = np.argmax(train_y, axis=1)
\end{lstlisting}
\end{itemize}.

\section{Parti della chatbot}
\subsection{Pretrain}
[AGGIUNGERE Immagine Codice]


Il codice rappresenta un processo per creare una chatbot basato su machine learning. Si avvale di BERT per generare rappresentazioni numeriche (embedding) delle frasi e poi costruisce un modello di classificazione per determinare i tag delle frasi in input. 
Il modello creato presenta tre strati: 
\begin{itemize}
	\item \textbf{Input}:  La dimensione degli embedding di BERT.
	\item \textbf{Strati nascosti}: Due strati densi con ReLU e Dropout per evitare overfitting.
	\item \textbf{Output}: Softmax per classificazione multi-classe.
\end{itemize}.

[AGGIUNGERE Immagine Codice]

\subsection{Analyze intents}
Il codice carica un file JSON che contiene gli intenti di un chatbot, li analizza e restituisce informazioni utili. Inizia leggendo il file e verificando che contenga la chiave "intents". Dopodiché, conta il numero totale di tag presenti, li elenca e analizza ciascun tag. Per ogni intento, stampa il nome del tag, il numero di frasi esempio associate (patterns) e il numero di risposte definite (responses). Se il file non viene trovato o non è un JSON valido, restituisce un messaggio d'errore.
\subsection{Chatbot interface}
Questo codice descrive un chatbot avanzato che combina l'uso di un modello di classificazione pre-addestrato con l'intelligenza artificiale generativa di Gemini. Il funzionamento si basa su diverse fasi. Per prima cosa, il sistema configura Gemini, un modello AI generativo che può creare risposte dinamiche e contestuali. Viene poi caricato il file intents.json, che contiene una struttura di intenti predefiniti, con frasi esempio e risposte associate, insieme al modello di classificazione e alla lista dei tag.\\
Quando un utente invia un messaggio, la frase viene elaborata utilizzando il modello BERT, che trasforma il testo in un vettore numerico (embedding) per analizzarne il significato. Questo embedding viene passato al modello di classificazione, che prevede a quale intento (tag) il messaggio corrisponde. Se il modello è sicuro della sua previsione, restituisce una risposta predefinita collegata al tag identificato. Se invece la confidenza è bassa, il chatbot utilizza Gemini per generare una risposta più flessibile e creativa, basandosi anche sul contesto della conversazione grazie alla cronologia memorizzata.\\
Tutta questa logica è integrata in una pipeline che decide dinamicamente se utilizzare risposte predefinite o dinamiche. Nel frattempo, la cronologia della chat viene costantemente aggiornata per garantire che ogni messaggio tenga conto delle interazioni precedenti. Infine, il chatbot funziona come un sistema interattivo: attende l'input dell'utente, genera una risposta adeguata e la presenta, aggiornando il contesto per i messaggi futuri. Questo approccio ibrido rende il chatbot capace di rispondere in modo sia preciso sia naturale, adattandosi alle diverse situazioni conversazionali.

\section{FAQ}
\subsection{Come posso configurare la chatbot nel mio ambiente locale?}


\newpage
\section{Glossario}
\begin{longtable}{|l|p{10cm}|}
	\hline
	\textbf{Termini} & \textbf{Descrizione} \\
	\hline
	\endfirsthead
	% Prima pagina della tabella (intestazione)
	\hline
	\textbf{Termini} & \textbf{Descrizione} \\
	\hline
	\endhead
	% Pagine successive della tabella (intestazione)
	\hline
	\endfoot
	% Fine della tabella
	\hline
	\endlastfoot
	
	% Contenuto della tabella
	Diversify & Progetto di Ingegneria del Software che prevede la realizzazione di un sito web per combattere le discriminazioni. \\
	\hline
	Gemini & Modello di linguaggio multimodale sviluppato da Google DeepMind. \\
	\hline
	PEAS & Performance Measure, Environment, Actuators, Sensors. \\
	\hline
	Bias & Errore sistematico causato da dati di addestramento sbilanciati o parziali, che porta il modello a produrre risultati distorti o discriminatori. Può influenzare l'equità, l'accuratezza e l'affidabilità delle previsioni. \\
	\hline
	NLP & Natural Language Processing. \\
	\hline
	Kaggle & Piattaforma online per data science e machine learning che offre competizioni, dataset pubblici, e strumenti collaborativi per sviluppare modelli e analizzare dati. \\
	\hline
	ChatGPT & Modello di linguaggio basato sull'architettura GPT (Generative Pre-trained Transformer), progettato per comprendere e generare testo in linguaggio naturale. \\
	\hline
	Tag & Etichetta che identifica un argomento specifico. \\
	\hline
	Pattern & Modelli o frasi che il bot cerca di riconoscere, spesso usando espressioni regolari o parole chiave. \\
	\hline
	Responses & Risposte predefinite che il bot restituisce quando un pattern corrisponde a un tag. \\
	\hline
	Intent & Gruppo formato da un tag, patterns e responses. \\
	\hline
	Dataset & Raccolta di dati organizzati in una struttura definita. \\
	\hline
	Numpy & Libreria Python fondamentale per il calcolo scientifico. \\
	\hline
	JSON & Formato di scambio dati leggero e leggibile. Allo stesso momento, è una libreria Python che permette la manipolazione di questi ultimi. \\
	\hline
	Pickle & Libreria Python che permette di serializzare (salvare) e deserializzare (caricare) oggetti Python in un formato binario. \\
	\hline
	Transformers & Libreria sviluppata da Hugging Face che semplifica l'uso di modelli avanzati di deep learning. \\
	\hline
	PyTorch & Libreria di machine learning, sviluppata da Facebook, usata per l'addrestramento di reti neurali. \\
	\hline
	TensorFlow/Keras & Libreria di machine learning open-source sviluppata da Google, utilizzata per costruire e addestrare modelli di deep learning. \\
	\hline
	Pretrain & Processo di addestramento iniziale di un modello di machine learning su un ampio dataset generico. \\
	\hline
	BERT & Bidirectional Encoder Representations from Transformers. Modello di deep learning per il trattamento del linguaggio naturale sviluppato da Google. \\
	\hline
	Embedding & Tecnica di rappresentazione numerica di parole, frasi o altre unità di dati in uno spazio continuo. \\
	\hline
\end{longtable}

\end{document}