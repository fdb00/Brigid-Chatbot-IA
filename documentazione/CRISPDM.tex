%%%%%%% INIZIO PREAMBOLO %%%%%%%

\documentclass[12pt, letterpaper]{article}
\usepackage[utf8]{inputenc}
\usepackage[T1]{fontenc}
\usepackage{graphicx}
\usepackage{titling}
\usepackage{geometry}
\usepackage{imakeidx}
\usepackage[italian]{babel}
\makeindex
\usepackage{listings}
\usepackage{xcolor}
\usepackage{caption}
\DeclareCaptionFormat{myformat}{\textbf{#1#2}#3} % Formato personalizzato
\captionsetup[lstlisting]{format=myformat, labelsep=space}
\renewcommand{\lstlistingname}{Codice} 

\lstset{
	basicstyle=\ttfamily\normalsize,    % Font monospace con dimensione normale
	keywordstyle=\color{teal}\bfseries, % Parole chiave in teal e grassetto
	commentstyle=\color{gray}\itshape,  % Commenti in grigio e corsivo
	stringstyle=\color{orange},         % Stringhe in arancione
	numbers=left,                       % Numerazione delle righe a sinistra
	numberstyle=\small\color{gray},     % Numeri di riga piccoli e grigi
	stepnumber=1,                       % Numerazione per ogni riga
	breaklines=true,                    % A capo automatico per righe lunghe
	frame=rounded,                      % Cornice arrotondata intorno al codice
	rulecolor=\color{lightgray},        % Colore del bordo
	backgroundcolor=\color{gray!10},    % Sfondo chiaro
	captionpos=b,                       % Posizione della didascalia (sotto il codice)
	showstringspaces=false,             % Non mostra spazi nelle stringhe
	tabsize=4,                          % Dimensione dei tabulatori
	xleftmargin=20pt,                   % Margine interno a sinistra
	xrightmargin=20pt,                  % Margine interno a destra
}

% Imposta i margini della pagina
\geometry{top=2cm, bottom=2cm, left=2.5cm, right=2.5cm}

% Dati per il titolo
\title{\Huge \textbf{Documentazione}\\[0.5cm]
\Large \textbf{Progetto Brigid AI Chatbot}}
\author{\large Catello Martone, Davide Viola, Gabriella Fede, Pasquale Anatriello}

%%%%%%% FINE PREAMBOLO %%%%%%%


\begin{document}
	
% Aggiungi il logo (opzionale)
\begin{figure}[t]
\centering
\includegraphics[width=0.3\textwidth]{brigid.png}
\end{figure}
	
% Genera il titolo
\maketitle
\vfill
% Informazioni aggiuntive
\begin{center}
Prof. Fabio Palomba, Corso di Fondamenti di Intelligenza Artificiale\\
Università degli Studi di Salerno, A.A. 2024/2025
\end{center}
	
\newpage
\tableofcontents
\newpage
	
\section{Esempio da cancellare}
Questo è un esempio di documento con un indice analitico.
	
\begin{lstlisting}[language=Python, caption={Hello World!}]
def hello_world():
print("Hello, World!")
\end{lstlisting}
	
\section{Introduzione: Analisi del problema}
La metodologia utilizzata è descritta qui.
\subsection{Forza Napoli}
Questa è una sottosezione.
	
\subsubsection{Dovrebbe essere una sottosottosezione}
provaaaaaa
	
\section{Prova}
La metodologia utilizzata è descritta qui.
	
\end{document}